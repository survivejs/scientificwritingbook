Paragraphs are the basic building block of your writing as they allow you to get your ideas by using sentences of text.
Typically a single paragraph consists of a theme defined in its first sentence, a body which may be multiple sentences after a theme, and a rheme which is the last sentence of a paragraph and concludes it \citep{rustipa2010theme}.

\section{How to construct paragraphs}

Although paragraphs sound simple, they are not always simple to write.
Simultaneously, even a long paragraph can be legible for the reader as long as you take care to connect the ideas well.
A basic way of connecting is to rely on repetition, and occasionally synonyms.
By repeating key words, you connect sentences together.

A good way to test how well your sentences connect is to split up a paragraph into separate sentences and then give it for someone else to reconstruct.
In case the other person was able to reconstruct your original paragraph, then the sentences of the paragraph flow naturally to each other.

\section{Types of paragraphs}

Based on my experience, paragraphs can be categorized roughly to the following types: meta, bridging, supporting, and concluding.

\begin{description}
    \item[Meta] paragraphs allow you to signal the reader about what is to come to make it easier to digest the content and tell about the connections within.
    \item[Bridging] paragraphs connect a section to another topically to allow for an easy transition to the reader and usually you find these at the end of a section should they exist.
    \item[Supporting] paragraphs develop the content and are the bread and butter of your work as this is where the main content exists.
    \item[Concluding] paragraphs capture and condense information so it is easier for the reader to understand your main points.
\end{description}

The idea is self-similar in sense that you can see similar structures at a different level of a scientific paper.
When you evaluate your work, it is worth thinking about what are the specific functions in place and what might be missing.

\section{Academic writing moves}

As you read academic papers, you might notice they have specific ways of describing things.
That is not by a coincidence but rather by convention.
In an academic writing style, you have a set of moves \citep{aull2020students} you can use to get your points through.

While reading papers, note how the authors use the language for signaling and getting their point through.
You might notice that academic writing tends to use hedging\footnote{Consider hedging as a way to describe uncertainty by the choice of words.} while using specific ways to lead the reader through the content as original thoughts of the authors are mixed with carefully chosen references that are put within a context.
The way you perform these operations is at the core of academic writing and it is academic writing moves that let you perform them fluently.

\subsection{Academic Phrasebank}

Although you can accumulate these writing moves over time by reading and writing your papers, it can be helpful to have a look at \href{https://www.phrasebank.manchester.ac.uk/}{Academic Phrasebank}\footnote{\url{https://www.phrasebank.manchester.ac.uk/}} maintained by the University of Manchester as they maintain lists of common phrases categorized for specific purposes.
Often there are multiple ways to say the same thing and then it is up to you to consider what makes the most sense for your purpose.

\subsection{Patterns of Organization}

Writing scientific papers in English is straightforward because there are specific patterns to follow.
Therefore, it is sensible to learn these patterns and a service called \href{https://www.ereadingworksheets.com/text-structure/patterns-of-organization/}{Patterns of Organization}\footnote{\url{https://www.ereadingworksheets.com/text-structure/patterns-of-organization/}} is the perfect starting point for picking up the basic ideas.
Once you have become familiar with the patterns, you will apply them in your work fluently, and it will become easier to put your ideas on paper (pun intended).

\section{Repetition and the importance of being explicit}

When editing, I tend to keep an eye on paragraph-level repetition of sentences (i.e., "However" or similar contrasting words) and add variance to make the text.
Simultaneously I take care to repeat keywords to make it easier to follow who is doing and what.
By leaving concepts implicit, you leave space for interpretation, and as you know, if there are more than one way to interpret then it is likely that the wrong option is chosen at least by some readers.
For this reason, I take care to eliminate words like "This" from my paragraphs in favor of being explicit about what is being discussed.

\section{When to use active form over passive}

When I was starting my studies, it was common in academia to favor passive form over active to hide the author and to be more objective in your presentation.
Interestingly enough, this seems to have changed over time as active form has become preferable \citep{ryba2021better}.

I tend to use both active and passive forms in my writing to make it clear what is my own thinking.
I prefer to use active form in chapters like Introduction, Methods, Discussion, and Conclusion, as those are natural places where to inject your own thinking and doing.

I think passive forms still have their place in writing, but it is something you should be mindful about when authoring.
When there is a clear actor, then I would avoid a passive form and instead declare the actor so that they are not hidden from the reader.

\section{Conclusion}

Paragraphs and sentences are the most basic building blocks of a scientific paper. You can make your work easier by understanding how to construct paragraphs and sentences effectively. There are basic patterns, such as theme-content-rheme, to understand and it is worth considering the function your paragraphs serve.

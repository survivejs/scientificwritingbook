One of the most complex parts of scientific writing is coming up with a topic.
More specifically, this concerns a research question or questions you want to study.
A question might come from a practical problem, or even better, you might be able to find an open problem in related literature.
In either case, it is essential to document the motivation behind your work, as this will be needed as a justification for your paper or thesis.

\section{Research question scopes your research}

Choosing your research question well is essential because it scopes your research.
Too broad or vague research questions might not yield helpful enough answers.
To give you a concrete example of a research question from one my papers \citep{vepsalainen2023implications}, consider the following: What are the technical opportunities and challenges of edge computing for static website hosting? To dissect the research question of my paper further, I have split it up below into separate components to consider:

\begin{description}
    \item[What are] - The question defines the type of the study. For example, here we are looking for a group of things.
    \item[the technical opportunities and challenges of] - More specifically we are looking for technical opportunities and challenges of something.
    \item[edge computing for static website hosting] - The final part of the question scopes the study to edge computing for static website hosting.
\end{description}

Although you might not be familiar with some of the technical wording within my research question, you can see how I composed the question for my paper.
More importantly the question covers the keywords of my papers and also gives some idea of what should be covered in the paper.
Especially those technical terms I had in my question should be visible in the background sections of my paper and likely they will be repeated across the paper given their importance.

\section{Fix your research question early}

As you can see based on the brief example above, there are clear benefits to thinking about your research question as it can help to position and scope your work.
When you are researching a topic you do not understand well yet, it is normal to start with a slightly vague question and then scope and adjust it over time.

\subsection{How many research questions to choose for a paper}

Occasionally splitting up a question to multiple ones can be beneficial and sometimes this may imply that you should focus on one of the subproblems to maintain the scope of your work while leaving rest of the problems for later.
The open problems should still be documented in the Conclusion section of your work at the end so it is clear to the reader that you have considered the broader context beyond your paper.

Given scientific writing is quite focused, it is important to fix your research question or questions early on although slight adjustment is allowed later.
It is natural to drop a question or two or to add modifiers to scope down the questions.
For short papers and BSc for example, often a single research question is enough.

\subsection{You know the least about the topic when you should know the most}

The fundamental problem related to choosing your research question is that you know the least at the beginning of a research process and the most at the end.
For this reason, it is essential to spend time with literature related to your research domain and take notes.
The fact that you know the least in the beginning has impact on which order you should write and we cover this problem in the Chapter \nameref{ch:elements}.

\subsection{Time spent understanding the topic saves time in the long run}

Spending time with the topic is valuable as it avoids you to arrive on a relevant research question that has not been explored in detail already.
The worst thing that could happen is that after performing your research you find out that someone else has published the same, or similar, results already.
You can largely avoid this risk by doing a diligent literature review before fixing on your research question.
For this reason, I recommend writing a personal literature review early on especially when you are new to your topic.

\subsection{Consider writing a personal literature review to explore the research space}

When starting and figuring out a topic, the difficulty is that you might not know enough about the topic to develop your own.
To train writing with low pressure, I recommend creating a personal literature review written scientifically, including sources.
This work does not have to be published anywhere but can be shown to your peers.
Most importantly, writing it will sharpen your thinking about the topic, and this will pay you back when writing the actual paper you intend to write.

A good literature review helps you to understand the currently open problems.
Ideally, you can motivate your problem based on these open problems and perhaps answer one or more.
That puts you in a strong position to give your work relevance.

\section{Deciding how to address the research question}

Your choice of a research question impacts the ways you can study the problem.
In my example above, I was looking for specific qualities to characterize technical opportunities and challenges and focused specifically on performance where I did a small quantitative study to understand a specific factor in detail.
Qualitative and quantitative research form the basic approaches and it is possible to combine them in a way like this.

% Generally, scientific research can be divided into qualitative or quantitative. In the qualitative approach, you want to specify the quality of something, for example, a sentiment, while quantitative deals with specific quantities. It is also possible to combine both within a single study to gain a complete view of a topic.

\subsection{Literature reviews and surveys are the easiest starting point}

The most straightforward type of study is a literature review or a survey where you address your research question based on existing material and then try to synthesize adequate answers.
Surveys are valuable because they can point out new, more specific research questions that can be addressed in future research efforts.
Surveying also has the benefit that it strengthens your understanding of what has been done already and helps you connect the dots between different papers as you see how they address your research question.
In short, surveys provide a more clear view of a topic while not necessarily producing a significant amount of new scientific information.
Due to their nature, surveys are a natural fit for the BSc level as they allow you to learn much about scientific writing, especially the first time.

\subsection{Empirical studies are more complex by nature}

Surveys can provide a good stepping stone towards empirical studies, where you might look into specific questions concretely, for example, by capturing user sentiments or benchmarking a particular property.
These studies are expected at the MSc level and industry as they provide specific insights into otherwise difficult-to-understand topics.

\subsection{Preferred research methods depend on your field}

It is difficult to give specific advice on researching your question as the exact methods and approaches tend to differ based on the field.
By getting familiar with papers within your field, you can quickly see common types of documents and research approaches and better understand what approach you can apply in your paper.

\section{Find good collaborators early on}

When you work in an academic context, collaboration is expected at some level although you might be doing most of the work alone.
For thesis work, you most likely work with a supervisor and an advisor, or multiple advisors, that communicate the expectations from the academic side and help you iterate on the content.
Given the impact of your collaborators, it is worth it to invest time into finding people that make sense given the scope and context of your work.
A good supervisor is able to help you on a high level of the process while good advisors can give concrete ideas on your topic and perhaps how to present it.

Although you could come up with your own research question, often researchers working in the academia have a group of questions of their own that they are happy to share with students.
Instead of understanding research space and its open problems, one way to get ahead is to work on a problem proposed by a researcher in your department.
Using a topic available from your department gives you a natural access to supporting people as you are working on something interesting to them automatically.

\section{Conclusion}

Choosing your research question well can have a significant impact on your work and it is important to invest time early on to discover a worthwhile problem as it is the starting point for your study.
Especially early on as you might not understand your specific field well, it can be worthwhile to write a rough literature review without any specific publication target as doing this will get you into the habit of capturing information in a scientific format while developing your understanding of what kind of research has been done already.

\citet{katz2009research} defines scientific articles as repositories of scientific observations including recipes to repeat them. I think that is a good description of what scientific writing is about as you capture results, share how you acquired them, and then discuss the implications of your findings.

\section{Scientific results should be presented in a clean manner}

Another important point that \citet{katz2009research} makes is that in a stereotypical paper, the results should be presented in a clean, clear, and unemotional form. That said, there has been a clear trend towards a more active form in scientific writing \citep{ryba2021better}, especially when you author sections, such as a part of the introduction or discussion.

\section{Scientific writing is about transmitting your results to the reader}

In \citet{gopen1990science}, the authors argue that science is often difficult to read, and likely the argument still holds today. The main point that \citet{gopen1990science} make in their article is that presentation matters and sloppy writing can obscure results and even fail to transmit vital information to the reader.

\section{Scientific writers translate results to a diverse set of readers}

In essence, scientific writing is about transmitting your results to the reader in a way they can understand. The challenge is that readers come from different backgrounds and have different levels of understanding. Your task as a writer is to consider this problem and you act as a translator between your results and the reader.

\section{Good practices and conventions are there to help}

Fortunately you do not have to start from scratch as there are good practices that have formed over time. For example, many readers expect to find a variant of so-called IMRAD structure covered in \nameref{ch:elements} and following the structure gives your text a degree of readability out of the box. There is more to scientific writing of course and you will learn more about making your text readable as you go through this book.

\section{How does scientific writing differ from other types of authoring}

As mentioned above, scientific writing has a strong focus on transmitting your results to the reader. That said, there is a lot you can do within this main constraint as a part of your work could be for example speculative as you are looking for good research problems to solve while a part of your work might include concrete measurement results.

Compared to other types of writing, you might notice that scientific literature uses so-called hedging words to give sentences a degree of uncertainty as few things are absolutely certain in science. A scientist is able to tell something about their degree of certainty through their use of words.

Another important aspect of scientific writing is that it often cannot exist in a vacuum. In other words, often you build up your writing on top of the work of others while pointing to the work of others through citations. Citing not only shows where your ideas are coming from, it also benefits other researchers in the field by allowing them to discover related material easier. In literature reviews, a good method is to examine references in detail to discover more papers to study to construct a good view of the research target.

\section{Conclusion}

The main task of scientific writing is to communicate your results to the reader in a clear manner. If you can achieve that, you have already achieved a lot. There are certain writing related specifics you should know and we will go through them in greater details in the following chapters so you become aware of good conventions and have a better idea of how to approach scientific writing.

In \citet{ryba2021better}, the authors measured the impact of authoring paper abstracts in an engaging, accessible style. \citet{ryba2021better} found out that their hypothesis was true and this modern style of writing improved the readers' readability, confidence, and understanding of the topic providing evidence that how you write matters.

\section{What is diversity in science}

It is good to consider that the readers of our papers might not be like us. They might come from diverse backgrounds and fields even.
As \citet{ryba2021better} point out, collaboration is needed to solve many of the big problems facing the modern society.
By taking care to author our papers in an accessible manner, we literally make our work open to new groups of people outside of our domain inviting new types of collaborations and progress.

\section{What is accessibility in science}

A good way to think about accessibility is what it is not.
Consider so-called old style of writing where the main aim of scientific paper is to be objective or even distant \citep{ryba2021better} and what that means for the reader.
Likely you have encountered these types of papers already.
Although the results might be interesting, it is likely you found the way the results were presented was in some way dull.

In papers written in an accessible way, the target is not to distance yourself from the topic but rather bring yourself to the work to some extent.
It is not only about bringing yourself to the work but also about the words you are using.
Many of the topics discussed in this book are about accessibility and that is not by a chance.
For example, the way we craft our titles is an important part of accessibility work and then same goes for summaries we provide for our readers to make it faster to scan our main ideas.

\section{Tables and figures benefit from accessible descriptions}

Accessibility does not end to text as often there are other types of media included to our papers.
Tables and figures are good examples of additional media that can benefit from accessibility work given you should provide descriptions to them.
Often journals encourage you to write accessible descriptions for tables and figures and this is particularly important for figures as it helps different groups of people to understand what your figures are about.
The main point is to cover the key ideas behind the figures in a few sentences\footnote{There is a good article related to digital accessibility from Harvard University at \url{https://accessibility.huit.harvard.edu/describe-content-images} that explains well how to describe images.}.

\section{Conclusion}

By keeping accessibility in mind, you make it easier for your readers to digest your work.
Considering accessibility improves the chances your work will be interpreted as you intend.
As a side effect of accessibility work, your writing becomes available beyond people that operate in your domain and thereby encourages cross-domain collaborations important for solving the most difficult problems facing our society.

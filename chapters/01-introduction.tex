% This writing guide has been tailored specifically for my students. It contains personal views, and you should not consider it to be the final word on the topic. Rather, I have captured my experience in this booklet after publishing a dozen scientific papers. In short, it is the kind of information that would have been useful for me to have before I started my efforts toward a doctorate.

% I want to give my special thanks to my students (too many to mention) for inspiring the work and to akx, tpr, Maurice Forget, and Eugene Woo for constructive feedback.

% \noindent\makebox[\linewidth]{\rule{\textwidth}{1pt}}

According to \citet{openWhatFirst}, the first scientific journal was published in 1665 but it is under debate whether it was a true journal or not.
The purpose of journals from early on was to work as a sort of news facility for scientists so they could learn from each other and create new science.
Therefore, it is not surprising journals still exist today, although other venues for sharing information have become available.

In this brief book, I want to go through my main observations related to scientific publishing specifically for those publishing in technical fields, such as computer science\footnote{It is likely that some of the advice given does not apply to fields, such as humanities, although I have taken care to include higher level topics.}.
The main target is to support my computer science students in their writing efforts, as I understood that I ended up giving similar advice to most and then forgot to give some.
As a side effect the existence of this brief book should save time and effort on both sides.
I know the university has its own courses and own material from which I have benefited a lot but there may be some value in capturing some of the insights in a condensed form to give a more personal view on the topic.
Although reading books about writing is useful, nothing beats practice.
If you can find a related course or a group, I recommend participating as it is highly useful to work with other people on your writing.

My path to writing this book was not the most straightforward.
I finished my MSc at the University of Jyväskylä in 2011 and then worked in the private sector until I was able to enroll in Aalto University's doctoral program in mid-2022.
During the time between, I started my own company (SurviveJS Oy) through which I published several technical books and did software consulting and related work. 
Therefore, my perspective on writing is not purely academic.
That said, scientific writing is an art of its own and although having experience in writing is beneficial, there is likely some adaptation to do.
Through this adaptation, you will become more well-rounded as a writer.

I have structured the book per theme, and technically, you could approach it in any order as it is more like a collection of related tips without a strong narrative.
I have outlined the contents briefly below:

\begin{enumerate}
    \item In \nameref{ch:what-is-scientific-writing}, I consider the special characteristics of scientific writing and show it is different from other types of authoring.
    \item In \nameref{ch:research-question}, I discuss the significance of choosing your research questions carefully as those shape the rest of the work.
    \item In \nameref{ch:elements}, I discuss the basic elements expected by a scientific paper.
    \item In \nameref{ch:process}, I discuss how I approach the process of writing scientific papers.
    \item In \nameref{ch:paragraphs}, I discuss how to construct paragraphs.
    \item In \nameref{ch:accessibility}, I discuss how to consider accessibility of your work.
    \item In \nameref{ch:citing}, I discuss how to cite accurately.
    \item In \nameref{ch:editing}, I discuss how to take your work from good to great through editing.
    \item In \nameref{ch:troubleshooting}, I consider how to deal with problems that you might inevitably face in your writing process.
    \item In \nameref{ch:technology}, I discuss tools I have found useful for writing scientific papers.
    \item In \nameref{ch:publishing}, I discuss publishing-related aspects in case you want to get your work out there.
    \item In \nameref{ch:popularizing}, I consider how to popularize your results and create a profile that is visible to other researchers.
    \item In \nameref{ch:resources}, I mention several resources where to find more information related to writing, as what you see here is only the tip of the iceberg.
    \item In \nameref{ap:latex} appendix, I go into my LaTeX typesetting system related recommendations.
\end{enumerate}

Remember that writing is a process and a journey and the more mileage you gain, the easier it gets.
Don't forget that also reading counts!

% I have been writing English for roughly two decades but I am still learning. I started self-publishing my own technical books around 2016 and was able to enroll to Aalto University's doctoral program mid-2022. Before that I finished my MSc at the University of Jyväskylä in 2011 and you could say I took a break from academia as the time did not feel right to go on with my studies then. Instead, I worked in the industry for roughly a decade mainly through my own company before putting one leg back to the academic side. It was only during my studies at Aalto that scientific publishing started to make sense.

% The web is the most prominent application platform globally, thanks to its vast user base. Although it started explicitly as a site platform in the 90s, it evolved into an application platform over time as its potential as such was recognized and interactive web applications became a reality. So-called single-page applications (SPAs) represent the current mainstream approach for developing complex web applications. While SPAs provide a good experience for developers, they come with a cost of their own for the users due to the technical assumptions underneath. Disappearing frameworks question these technical assumptions and allow developers to address user needs better while retaining the benefits of the earlier approaches.

% In this short book, I want to give you a quick overview of the latest developments in the field while explaining why I believe disappearing frameworks will inspire the shape of web development during the coming years. Many of the ideas are accessible already, and after reading this book, you will know where to look when evaluating new frameworks while being able to appreciate their level of innovation better. At the same time, you will see the current mainstream frameworks in a different light and understand their technical constraints better.

% We start by delving into the history of web development to understand what has motivated the development of disappearing frameworks and why it is such an important topic. You could say that disappearing frameworks emerge from the pressures of both users and developers as the threshold for what is expected from a web application rises each year while developers are expected to deliver faster. It is within this intersection where technical innovation occurs as new tooling can give a higher baseline enabling developers to deliver more quickly and more robust applications for their users.

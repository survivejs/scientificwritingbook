Although getting papers published and sharing your results is important, occasionally it is equally important to try to spread the word beyond your niche by popularizing your work.
Popularizing your work means wrapping it in another form that is easier to digest in some way.

\section{Create a public researcher profile}

It is possible to make your easier to discover by other scientists in your domain by maintaining a public profile.
A common way to do this is to set up an \href{https://orcid.org/}{ORCID}\footnote{\url{https://orcid.org/}} record that captures your employment, education, professional activities, and works.
Better yet, many papers allow you to attach ORCID identifier directly to them meaning the system will discover your work automatically and attach it to your profile.

\href{https://scholar.google.com/}{Google Scholar}\footnote{\url{https://scholar.google.com/}} creates researcher profiles automatically although it is a good idea to claim yours.
Compared to ORCID, Google Scholar goes a step further by showing your citation counts in a graphical manner while showing where your citations are coming from.

It can be a good idea to claim a \href{https://www.researchgate.net/}{ResearchGate}\footnote{\url{https://www.researchgate.net/}} profile as it helps with your discoverability.
ResearchGate allows you to upload files related to your research and I have used it as a host for my presentation slides, for example.

Beyond the options above, one way is to set up your own website where you curate your own research.
I have done this on my own website in a specific section\footnote{\url{https://survivejs.com/research/}} to highlight the papers I consider the most significant while also giving credit to my students.

\section{Measuring popularity of papers}

To hold an academic position, often a certain amount of publications and popularity are expected to show that you are contributing to the field.
Typically the popularity of papers is measured through citation counts \citep{zhou2012quantifying}, but so-called \href{https://www.altmetric.com/}{Altmetrics} represent an interesting option based on online engagement around your content.
Altmetrics give some clue how you might popularize your work as the concept implies that if you wrap your results into a format compatible with for example social media, you can reach more people so they are aware of your results.

Papers tend to require time to accumulate references and it is common that a specific paper might stand out while others fail to gain significant attention.
I believe this also has to do with your specific niche and the amount of research within the space.

\section{Create material aimed at specific markets}

Another way to look at popularization of your results is to consider authoring secondary material around your research.
Producing blog posts, videos, or other media may be good alternate means to condense and present your research in an easy to access form.
Alternate medias allow you to avoid typical constraints of academic publishing and focus on a specific type of reader.
Through focus, you can cut away unnecessary information and create a focused take on your results that resonates with your target and generates interest.

\section{Conclusion}

By creating a public profile and popularizing your work you not only make yourself more visible as a researcher but also allow your work to have greater impact.
Only work that can be easily discovered by other researchers and public can have impact and therefore it is worth it to put some thought into how to achieve this.
In technical terms there are many options and it is good to keep the constraints of the media and target in mind to shape your message appropriately.

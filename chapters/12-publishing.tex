Assuming you want to make your work available to a larger public, you must consider where to publish.
Conversely, you might first work out a publishing target and then figure out what it takes to publish there, assuming you are working towards a doctorate.

\section{Types of papers}

There are several types of papers you will likely write during your career as a scientific writer:

\begin{description}
    \item[Essay] on a given topic to provide a personal viewpoint, for example. This is the most personal and, by nature, the most loose.
    \item[Conference paper] showcases early results to a limited audience to later grow it into a journal paper. Conference papers are evaluated by conference reviewers, and based on their reviews, you might or might not get into the conference to present your work. Sometimes conference papers might include major results already making them valuable publications. Typically conference papers have strict page limits which can be occasionally a challenge as this may force you to condense your content to fit.
    \item[Journal paper] goes into deeper detail about a topic as page limits tend to be relaxed. Generally, journal papers can be more challenging to get through, especially in highly rated journals, as they go through a strict process, including multiple reviews.
    \item[Thesis] captures knowledge on a specific topic, and the scope depends on the educational level, and the same goes for the rigor related to the review. At Bachelor's level, these are considered more about training writing, while at higher levels, more is expected from the results. However, even a Bachelor's work can occasionally provide novel insights into a topic and have value to other researchers. Because of the fact that Bachelor's theses are usually the first publications of their authors, often universities discourage using them as direct sources although exceptions can be made if the thesis quality is high enough.
\end{description}

% Essays are the easiest to publish, as you could publish them through your personal sources or online services with a low threshold. For these, you have an institute that will publish them by definition, assuming we are talking about academic work. Conference and journal papers are more involved as they consider a specific context, and I will go into further details related to publishing these in the next sections.

\section{Where and how to publish?}

When working on a thesis related to your studies, the answer to where to publish is clear, as your institution will handle the problem.
Publishing can become a problem when you are working towards a PhD, however, mainly if you are composing your thesis out of separate publications.

\subsection{Refer to senior colleagues for ideas on where to publish}

Especially when starting, it is worth referring to your senior colleagues as they will have a good idea of venues available within your niche.
The amount of conferences available depends on the popularity of your topic and the quality tends to vary accordingly.

After you have a good idea of what are the main conferences within your niche, you should see what kind of papers they have published during the past few years and look at especially their best papers.
By doing this scouting step, you will understand better what kind of papers the conferences tend to accept and what kind of topics they cover as this helps you to shape the way you present your work.

Generally put, an excellent way to work is to aim for a conference paper and then develop it into a full journal paper that can stand on its own.
Occasionally, you can skip the conference step and go directly for a journal.

\subsection{Consider using search engines to find publishers}

As mentioned in the \nameref{ch:technology} chapter, most likely there's a search, such as Jufo\footnote{Jufo (\url{https://www.tsv.fi/julkaisufoorumi/haku.php?lang=en}) is a Finnish search for potential venues and it is generally a bad sign if your potential target does not exist there.}, available to you to help figure out the prestige of a publishing venue.

\subsection{Take care to avoid predatory publishers}

You should do everything you can to avoid so-called predatory journals as they will waste your effort and make you, or your institution, pay for it.
You will likely begin to receive emails from predatory journals after publishing your first papers and often the emails look official.
Usually the name of a predatory journal looks like an official one but it is misleading on some level.
Examining the content, editors, and published work of a journal helps to understand you better if a journal is suspicious one.
I recommend using \href{https://beallslist.net/}{Beall's List}\footnote{\url{https://beallslist.net/}} to see if the journal in question is a predatory one or not.

To make things worse, predatory companies have extended their business to predatory academic conferences.
Predatory conferences share similar caveats as predatory journals but go a step further in their dishonesty.
\href{https://touromed.libguides.com/c.php?g=829080}{Touro College of Pharmacy}\footnote{\url{https://touromed.libguides.com/c.php?g=829080}} has written a good article about the topic describing characteristics of predatory conferences in detail.

A good tactic to figure out good publishers or conferences is to look at your references and see where they published.
Depending on the field, you might find a couple of highly rated conferences or journals and perhaps multiple less prestigious ones.

\href{https://thinkchecksubmit.org/}{Think, Check, Submit}\footnote{\url{https://thinkchecksubmit.org/}} goes into greater detail on predatory publishing and you can find good checklists on the website to validate journals.

\subsection{Consider writing with specific reviewers in mind}

Although scientific papers are written for the scientific community, the papers are reviewed by individual people.
Given conferences are open about who their reviewers are, you can use this information to your benefit by researching the work of the reviewers as often these people have a good standing in the community.
By doing your homework, you can address their findings in a subtle way in your paper should they work topic-wise.
It is a nice additional touch to acknowledge the work of the committee and show that you understand the topic well.

\subsection{Consider making your work more broadly available}

After your work has been initially published, it may be a worthwhile idea to make your available through platforms like \href{https://www.researchgate.net/}{Research\-Gate}\footnote{\url{https://www.researchgate.net/}} or \href{https://arxiv.org/}{arXiv}\footnote{\url{https://arxiv.org/}}.
Before publishing through other sources than your primary publisher, you should make sure they permit sharing the work elsewhere.

arXiv is particularly suitable for publishing your accepted manuscripts early, assuming you have the permission of the publisher.
Often, this is allowed as long as you link to the publication appropriately, and doing this enables you to get the work out there faster, although arXiv may require a while to review your submission.

Assuming you want to publish your thesis through arXiv or a similar platform, it is a good idea to wait until grading is complete to avoid issues during the grading process.

\section{Consider publishing speed}

When trying to publish, it is good to consider the speed in which different outlets operate.
Obviously self-publishing is the fastest way to publish as in that case you are in control, but in this case your work won't be peer reviewed and does not come with the stamp of some level of approval from your scientific peers.
Assuming you can get your paper approved to a conference, conferences tend to publish fast and often they allow publishing accepted manuscripts early assuming your include necessary references to the conference as most likely they will hold the copyright of your work.
Journals can take the longest to publish especially if you are aiming for a reputable one although exceptions, such as \href{https://ieeeaccess.ieee.org/}{IEEE Access}\footnote{\url{https://ieeeaccess.ieee.org/}}, exist as they provide a fast track for journal papers that are in a long format.

In practice, journals can require at least months to publish assuming they accept your paper.
For this reason, it can be a worthwhile to work on multiple research directions at once to use the waiting time in a productive manner.

\section{Feedback and how to get it}

Although you could go and complete a piece on your own, most likely, you will eventually have readers.
The sooner you can get others involved with your work, the better.
Initially, it might be your advisors, but your circle of initial readers could also include people from outside, depending on what kind of feedback you are looking for.

\subsection{Show your rough ideas to your advisors}

Early on, it is an excellent idea to produce plenty of crude text and put your early thoughts on paper so that your advisors can see where you are going on a high level.
As your ideas formalize, the work will become more solid.
Working loosely early on allows fast adjustment and avoids waste. However, some are unavoidable parts of the effort, mainly when working on a topic you do not know well.

\subsection{Bring more people on board as the work is stable}

As your work feels more solid, it is a good idea to get more eyes on it.
Then, you will gain feedback from a fresh perspective from people who have not been exposed to your writing yet.
One of the most complex parts of writing is becoming blind to your own assumptions, and external feedback outside of your inner circle can help to address this point.
One particular form of feedback is so-called review feedback, and I will discuss the topic in detail in the next brief section.

\subsection{Participate in student workshops and seminars}

Universities and conferences often organize topic-specific workshops and seminars where people interested in a specific topic can get together to share their knowledge regardless of their level although often these are targeted specifically at students.
The good side about participating in public events like these is that it allows you to get feedback on your early ideas fast and help refine them further.
You will also develop connections you might need later on and there might be potential collaborators in the crowd.

\section{On dealing with review feedback}

A part of academic publishing involves dealing with review feedback, as often journals follow a double-blind review procedure or a similar process. When submitting, occasionally, you may also have to anonymize the paper sufficiently to avoid bias.
You will find that each venue has its rules, and it is imperative to follow them to prevent an instant rejection of technicality.

\subsection{Especially first submission is unpredictable}

When developing a paper, it is good to remember that not every paper will make it through on the first attempt, although you can improve your chances through a strong and fluent presentation of your content.
Occasionally, not even a fluent presentation is enough, as there may be something else to fix.

\subsection{How to interpret review feedback}

Typically review feedback is provided in multiple forms: a numeric rating and textual feedback.
A numeric rating might vary for example from -2 (strong reject) to +2 (strong accept) and condense the reviewer's recommendation as a single figure.
Textual feedback is the more interesting part and the way it is written tends to depend on the reviewer as some are terse while others can be elaborate.

When you look at the feedback, pay attention to repeating themes and consider if the reviewers understood your content and findings the way you expected.
Occasionally you can see from the wording that they perhaps missed out on some vital point or misinterpreted your content.
Sometimes the problem might have to do with the fit of the publication to the venue and although your content might have been well presented, it might still have missed the mark by not hitting the type of paper the publication expects.

\subsection{Remember that reviewers are there to help you}

Even though feedback you receive may initially feel crushing, do not take it personally.
Instead, consider it as a constructive way to develop your content further.
Often, the most critical feedback teaches you the most and allows you to do far better in your subsequent attempts.

Failure does not make you a bad writer, and most, if not all of us, have failed at some point.
Sometimes you may have to repackage your content or try a different direction, but don't give up.

\subsection{Responding to reviewer comments}

Assuming your work was accepted to a publication, it is likely that you will have to respond to reviewer comments as often acceptance relies on fixing weak points that were identified in your work.
When responding, it is a good idea to use cordial language and then briefly explain how you have fixed the issues in a revision.

\section{Reviewer perspective}

Academic reviews are often performed by your peers that do the work for free as a part of their other duties.
In other words these people can be quite overworked and might not have a lot of time to interpret your work and findings.
For this reason it is worth it polish the readability of your work and make sure your main points stand out so that they are easy to understand.

As you progress with your academic career, it is possible that you will get invited to become a reviewer for a publication or a conference.
Getting involved with review process may be a good idea as that gives you additional perspective on how reviews work and helps you to subsequently improve the quality of your own submissions as you have a better idea of what is expected from a good entry.

\subsection{Peer review process}

Note that review process tends to differ per publication and there is no single standard way to do it.
Regardless of the process, the purpose is the same and that is making sure that the accepted work fulfills publication standards and is considered worth publishing under the stamp of peer review.
Peer reviewing itself does not guarantee quality but it is one indicator of it and generally academic tend to favor peer reviewed sources for this reason.

\href{https://libguides.mssm.edu/peerreview/types}{Icahn School of Medicine at Mount Sinai}\footnote{\url{https://libguides.mssm.edu/peerreview/types}} has listed different types of peer review processes and it is a good resource to study if you want to know more about possible variants.
Generally processes tend to vary based on how much information about writer and reviewers is passed around.
% Also collaborative reviews are possible where the work is improved in collaboration of the writer and the reviewers.

\section{Giving credit}

Given most scientific publications include work from multiple parties, it is important to give credit to contributors.
Typically each field, and even publisher, has their own rules on how to do this and it is enough to follow the conventions.

An interesting way to give credit is to use so-called \href{https://credit.niso.org/}{CRediT taxonomy}\footnote{\url{https://credit.niso.org/}} at the end of your paper where the contribution type of each contributor has been made explicit. I have included a sample credit based on \citet{elsevierCRediTAuthor} below:

\begin{quote}
\textbf{Zhang San}: Conceptualization, Methodology, Software \textbf{Priya Singh}.: Data curation, Writing- Original draft preparation. \textbf{Wang Wu}: Visualization, Investigation. \textbf{Jan Jansen}: Supervision.: \textbf{Ajay Kumar}: Software, Validation.: \textbf{Sun Qi}: Writing- Reviewing and Editing,
\end{quote}

\section{Conclusion}

Publishing is an inevitable part of scientific writing especially if you are going for a higher level degree or want to share your results with other researchers and public.
For a BSc/MSc, you do not have to worry about publishing, though, and it becomes more important topic only if you pursue a doctorate.
Occasionally MSc results might be converted as a separate paper to publish, though, and even BSc theses may have interesting results that are worth sharing beyond the formally required steps.

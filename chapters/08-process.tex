The basic process behind scientific writing does not differ much from other types of writing.
The main difference is that, by its nature, it is more rigorous, and you have to maintain your references.

\section{Architects and gardeners}

According to \citet{grrmArchitectsGardeners2024}, there are two types of writers: architects and gardeners.
Architects tend to plan the structure of their content upfront while garderers grow the content first and develop the structure later.
Even gardeners have a rough idea of what they plan to build, but their approach is more organic.

In practice, good writers can use both approaches although it may be natural to use one general approach over another.
Even if you approach writing structure first, it is likely that you will improve your structure as you keep on working.
In the context of scientific writing, some of the structure is fortunately given so there is less work to do in structural sense although you do have to consider your subtitles carefully to make sure they support your thesis.

\section{Basic process}

I have outlined my basic process behind scientific writing below:

\begin{enumerate}
    \item Research the topic\footnote{Occasionally this is called as \href{https://en.wikipedia.org/wiki/Sensemaking}{sensemaking} (\url{https://en.wikipedia.org/wiki/Sensemaking}) given especially early on you are trying to figure out the key terms and relationships of a topic and document them. For this type of work, using techniques, like mindmapping, is useful.} and take notes
    \item Write
    \item Gather feedback and edit
\end{enumerate}

The process is cyclic in that you will discover smaller topics to understand in greater detail within your more extensive study.
In essence, research is about building a narrative and documenting your discoveries.
Research is about gathering information and presenting it so that it is digestible to other scientists and even people outside of your immediate field.

\section{Develop a habit of taking notes}

I have found it helpful to document fruitful Google Scholar search phrases in a file within my Overleaf projects.
Other related links and even notes related to specific references can exist there.
Additionally, tools like \href{https://www.zotero.org/}{Zotero}\footnote{\url{https://www.zotero.org/}} or \href{https://www.mendeley.com/}{Mendeley}\footnote{\url{https://www.mendeley.com/}} can be useful. % for tracking.

Learning to take good notes can be highly beneficial, especially when you are new to the research domain and still trying to figure out what to do.
I know it is time-consuming to go through papers, but at the same time, you will gain a more complete view of what has been done.
The work spent researching will save time spent writing.

\subsection{Notetaking methods and related tooling}

There are specific notetaking methods, such as \href{https://zettelkasten.de/overview/}{Zettelkasten}\footnote{\url{https://zettelkasten.de/overview/}}, and tooling, for example \href{https://obsidian.md/}{Obsidian}\footnote{\url{https://obsidian.md/}} or \href{https://logseq.com/}{Logseq}\footnote{\url{https://logseq.com/}}, that are worth exploring\footnote{See also \url{https://en.wikipedia.org/wiki/Personal_wiki}.}.
Figuring out a good note-taking process is worth the investment as you will spend much time simply taking notes and piecing ideas together.
It is worth your while to choose a couple of tools to support your workflow.

\subsection{Maintain a research diary}

I recommend maintaining a research diary to capture your observations and progress.
The same concept is commonly known also as a lab notebook and the idea is to put information from your memory to a permanent storage so that it is safe to resume your work without forgetting something essential \citep{samlab2020}\footnote{Jupyter maintains a set of tools useful for this type of work and you can find them at \url{https://jupyter.org/}.}.
It can be enough to have a single file in your project with paper related notes that includes essential information related to it, such as where you found it and what did you learn from the paper.
There are sophisticated systems for capturing research papers and related information and I will cover them in the \nameref{ch:technology} chapter.

Another, complementary approach is to pull essential points to your initial draft through short references as this helps you in compiling related information together so it is easy to combine the information to complete ideas.

\subsection{Capture progress to understand open questions}

Although it takes time, it can be worthwhile to document your progress, primarily if you work with mentors. In this practice familiar to software development, you treat each iteration of your work as a release to highlight what has changed since the previous one.
For example, if you meet with your advisors biweekly, you could capture the changes using bullet points to illustrate your progress to a document you maintain close to your document source files.

Another worthwhile practice is capturing questions that arise as you work so you don't forget to ask them when you meet with your advisors.
Your advisors may also be able to respond to the questions in text format to capture this information.

I have found both practices beneficial when working with students, as they make the process transparent and allow documenting progress clearly.

\section{Good writing order avoids waste}

I have found the following writing order conducive to my way of working:

\begin{enumerate}
    \item Do a rough pass on introduction. The most important thing is to set up a basic structure and fix the early research questions and research method as those will shape rest of the work.
    \item Set up initial chapters for the paper so that content can be attached to it. The content chapter naming and amount is likely to change at this point.
    \item Focus on the hardest part of the paper. In practice this means the research part that addresses the research questions.
    \item After there are initial results, consider what information the reader should know to understand them. Usually there are several concepts and perhaps some historical context that is beneficial to provide. I fill in the background chapters at this point while considering their naming carefully.
    \item Fill in Discussion chapter with insights gained during the research. I also consider the limitations of my work in discussion.
    \item Add Conclusion chapter while reminding the reader of the research questions, addressing the main findings, and considering open open research questions.
    \item Rewrite introduction based on the knowledge gained while putting the work in context.
    \item Write a short abstract based on the paper content.
\end{enumerate}

The key points are to let the hardest parts of the work to guide the background and to revisit introduction near the end of the process.
Likely there are other possible writing orders but I have found this one to work well for me as it provides a minimal amount of waste usually.

\subsection{Leave gaps to fill later}

A good technique is to leave clear gaps to your structure and text while marking what to research.
As you gain conceptual knowledge related to your topic, you will be able to further iterate on your structure and create more gaps to fill which you will address as you proceed on the paper.

\subsection{Be prepared to rewrite}

As you develop your material, it is natural to rewrite it reflect your improved understanding of the topic so do not get too attached to your output.
By nature, writing is an iterative process and when you submit your paper, you have decided that you have done enough iteration and your paper is ready for review.

% Although you could write the abstract first, the introduction then, and other content sections after, that is not always the most conducive order. I have found it helpful to do a loose pass on early sections first and then focus on more complex parts that require more work before doing another pass on the abstract and introduction. This is because abstract and introduction are the easiest ones to write only after you have close to finished other parts of your paper as they require the most evident mind.

% Another helpful technique is to leave gaps in the text and fill them with TODOs\footnote{LaTeX package called todonotes can be helpful for this purpose, and you can find it at \url{https://ctan.org/pkg/todonotes}.} and initial lists of things to expand on later. While revising, occasionally, I comment on old paragraphs and rewrite them completely. That way, I don't lose points I might want to use later.

\section{Good titles make your work readable}

Although figuring out titles are a small part of the overall effort of writing a paper, it is an important aspect of the work as titles are a promise of content.
The main purpose of titles is to scope your work while giving hints to the reader on what to expect.
An overly broad title promises too much and might give the reader a wrong impression.
A suitable chosen title scopes the work carefully while highlighting keywords making it easier to discover the paper.

\subsection{All titles matter}

Title-related work goes beyond the main title of the paper as suitable secondary titles, such as chapter, section, or even paragraph titles, can give additional meta-level structure to a paper making it easier to read, and perhaps more importantly, to write.
A title is a promise of what is to come and by putting some thought into your titles can go a long way in terms of giving your paper structure while improving legibility.

\subsection{On debugging titles}

A good way to debug titles is to have a look at your table of contents.
Doing this also gives you some idea of how you might improve the structure of your paper.
Occasionally a certain order makes more sense than another as chapters tend to build on top of each other.

In my work, my high-level titles are quite abstract although I tend to avoid using a "Background" chapter and instead try to figure out a more specific title while thinking about the narrative.

\section{Conclusion}

A good writing process allows you to make constant progress as you will always have something clear to do.
The main point is that, especially in scientific writing, reading is your fuel as understanding the related work by others helps you to develop your own work while putting it to a clear scientific context through references.

Although we write to capture our findings, it is important to consider our readers.
For this reason, it is important to take the time to gather feedback and edit your work for readability as editing work can help to elevate a good paper to a great one.
By making it clear to your readers what the results are about, you also make it more likely for your papers to be received positively.

It is good to acknowledge as a writer where you stand in the architect and gardener axis.
I believe good writers can use both mindsets, but it is natural to be aligned towards one end and then understand the benefits and drawbacks to that.
The key thing is to discover a process that works for you.

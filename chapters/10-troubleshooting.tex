Writing is a muscle.
The more you train, the more effortless it becomes to use it.
Writing is not only about writing, though, as you expand your vocabulary and gain more tools by reading and paying attention to solutions that work.
It is equally important to develop a taste to be able to tell what kind of writing you want to avoid.

\section{Writing is a mileage game}

One way to characterize writing is that it is a way of thinking out loud.
By forcing yourself to put words in a meaningful sequence on paper, you force yourself to turn your thoughts into a hopefully understandable form to your readers.
Putting words on paper is only half of the story, as thoughts become more apparent over time as you understand your topic better.
That is where editing comes in, as it allows you to clarify your message and how you convey it to your readers.

Writing tends to be a mileage game because it gets easier the more you read and write.
It is an act of thinking out loud but on paper and formalizing my thoughts.
By writing, I can learn a lot about a topic as it forces me to highlight what I don't know and to research.
You can think of writing as a way of learning and documenting what you have learned to others.
Most importantly writing helps you to sharpen your understanding about a specific topic.

\section{Getting unstuck}

It is almost inevitable that you will have tougher times during your writing process. According to \citet{rose2009writer}, writer's block can be described as "“the inability to begin or continue writing for reasons other than a lack of basic skill or commitment".
The question is, how to unblock yourself and in part this comes to understanding yourself.
If you notice that writing is difficult, perhaps there is something else you can do meanwhile.
For example, if you are running out of ideas it can be a good idea to delve into papers and find inspiration elsewhere.
Sometimes it can be beneficial to spend a week or two away from a manuscript in progress to gain perspective on how to improve it.

I think one of the key techniques in avoiding a writer's block is to develop a good process and document a realistic timeline as then you know what has to be done and when.
If you work with an advisor, then figuring out good ways of collaboration is beneficial. I have noticed a biweekly interval of personal feedback works well for the most while others prefer more asynchronous approach.
Regardless of your approach, I recommend constant progress over sudden progress near delivery deadlines.
Although latter can work, it is also the more stressful way to work and it does not give you time to reflect.

\section{Power of summarization}

A classic wisdom related to scientific authoring is that if you do not know what to write, create a table or an image instead.
When you start thinking about the relationships between concepts this way, it might eventually unblock your writing.
Tables and figures achieve this task well as they force you to approach the problem from a new angle.

Another technique is to create lists to capture key points.
You should not overuse them, however, and often lists are a good intermediate form before conversion to prose.
One, perhaps underused form of lists, is called a definition list and usually writing environments have a special way of modeling it as it will render differently than the usual ordered or unordered lists.

\section{From chaos to order over time}

In general, producing a scientific paper is akin to going from a chaotic situation to an orderly one as you try to discover related research to compose a coherent whole.
A part of the work is creative and even chance-based, as it is not always easy to find good papers to support your work.
Sometimes, the hard part is figuring out the correct keywords to use in a search, and this can be particularly hard in fields where the terms used to signify the same things change over time as concepts are rediscovered.

Another way to look at scientific writing is that it is similar to developing an analog photo where you start from a dark room and several ingredients to make the photo turn into something understandable.
Producing a paper understandable to other parties is your main task, and you should focus on readability to make sure your points get through in the way you intend.
Occasionally, this can be difficult due to space constraints, and it may require several rewrites to package the message in a way that captures the essential information while being understandable in a non-ambiguous manner.

\section{Leave your work to an interesting spot to continue}

Given writing is closer to a marathon than a sprint, it takes a certain amount of perseverance to finish.
A classic trick is to leave your work to a spot where it is interesting to continue next time.
What will happen is that given you did not finish, likely your subconsciousness will keep on working on the part so it is easier write once you return to your writing.

\section{Conclusion}

Starting to write scientific papers can be difficult especially early on.
Assuming English is not your mother tongue, you have additional hurdles to overcome as writing tradition might be different than what you are familiar with. 
That said, scientific writing rewards a degree of persistency and it is more about the general approach than writing itself.
Although writing is an important part of the work, it is only a part of the big picture.
In case you get stuck with writing, likely there are other ways how you can take your research forward for a moment.

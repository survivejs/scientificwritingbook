Scientific writing in particular comes with a strong attribution requirement as it is vital to show where ideas originate from.
The general rule is that know facts can be stated without a reference, but for anything specific you most likely want to include a source.
Including sources shows that you have studied the topic and give respect to earlier work done in the field.
Proper sourcing also allows you to discuss with prior work and bring up your points in contrast to theirs.

\section{Types of citations}

The exact way how you should cite depends on your field and journals usually have an opinion or two related to formatting.
Generally put, I use the following ways in my work: textual, parenthetical, direct, and combined citations.
I will cover these types in the next sections in greater detail.

\subsection{Textual citation}

Textual citation is useful for constructing a sentence based on a citation as in the following example: In \citet{vepsalainen2023implications}, the authors noted that edge computing can provide performance benefits for static website hosting while techniques, such as ISR, can be used to address the build cost of SSG.

\subsection{Parenthetical citation}

Parenthetical citation can be used for integrating ideas from papers within a sentence as in the following example: In our study, we found out that edge computing has strong benefits when combined with SSG confirming the observations of \citep{vepsalainen2023implications}.

\subsection{Direct citation}

Direct citation is useful when you want to capture exact wording as in the following example: In \citet{vepsalainen2023implications}, the authors stated that "...there are still open questions related to techniques, their applicability in other environments, and their limitations.". Often you do not have to use this form, however, and paraphrasing may be the recommended way to go.

\subsection{Combined citation}

Combined citations can be used when you want to bring the same observation from multiple source together to strengthen a point as in the following example: It seems there is a strong demand for new web technologies that address performance demands of the modern web as implied by \citep{vepsalainen2023implications, vepsalainen2023rise}.

\section{Cite per sentence, not per paragraph}

Regardless of the citation style you prefer, I have found that it is better to cite per sentence over paragraph-level citations as that makes it clearer to me what is my own thinking and what is coming from elsewhere.
The risk with paragraph-level citations is that even though they might be initially correct, they might become inaccurate over time as you work on your paragraphs.
Sentence-level citations may lead to repetitive looking text but you could also consider this as a reminder to do more research to pull information from more diverse sources to add credibility to your text.
If you are interested in technical details, I cover LaTeX-specific citation syntax in Section \nameref{sec:citing-in-latex}.

% It is difficult to give exact rules on how to cite as that depends on your field.
% I tend to favor sentence-level citations in my work and avoid general paragraph style entirely as for me that is unclear and invites trouble, especially if you include your own ideas within the paragraph.

% The great benefit of using LaTeX and BibTeX is that the combination allows you to choose the style through configuration although you still have to take care to cite accurately within your text.


\section{Conclusion}

Citing correctly is an essential part of scientific writing and you should take care to learn to cite.
Citing can be slow work especially early on, but as you progress as a writer it becomes easier to integrate ideas from other people to your work and it becomes natural over time.
Citations allow us to stand on the shoulders of giants and respect the work that has been done before us while contributing to the scientific corpus of knowledge to continue the tradition.

In case you want to learn more about citing, \href{https://writing.wisc.edu/handbook/documentation/}{University of Wisconsin-Madison's writing guide}\footnote{\url{https://writing.wisc.edu/handbook/documentation/}} covers different styles adequately.

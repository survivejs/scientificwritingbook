After you have finished your initial draft with essential pieces in place, you might notice the work is still a little rough.
The purpose of editing is to take a piece and improve it by some metric, such as readability.
Occasionally you might have to edit for other factors to fit publication limits, for example length.
You could also edit for tone to make your work sound more scientific or you might edit passive sections to become more active to bring your own voice to play.

\section{Detaching yourself from the work}

As you develop your work, you will become intimately familiar with your subject.
The problem is that understanding your topic well makes it difficult for you to spot possible gaps in your explanations.
An easy solution is to put the work away for a while and get back to it as this might reveal new issues.
Understandably this isn't always possible as there might be deadlines to meet.

Another option could be to approach the work to edit from the end as this detaches you from the structure naturally and allows you to edit effectively per paragraph.

\section{Get external people to read your work}

It is a good idea to get several external people to read your paper as they will have a fresh perspective on the content and they will be able to provide feedback related to how you introduce the topic and your results.
Besides feedback from other people, I leverage tools like Grammarly to capture easy mistakes and I discuss these kind of tools in the \nameref{ch:technology} chapter.

\section{Edit for section level clarity}

To edit for section level clarity, I go through my table of contents\footnote{In case your work does not have, or need, a table of contents, it can still be a good idea to generate for editing purposes as that gives you a good overview of your paper. If you do not have to deliver one, you can leave it out from the final version.}, making sure content flows well together topically, and that the work has been appropriately paced.
What might happen occasionally is that a chapter begins to live a life of its own and grow to dominate your paper.
In that case, it is good to consider if all the content is absolutely needed or if the chapter could be split up somehow.
Occasionally certain ordering makes more sense than another as then you can rely on earlier concepts and therefore perhaps avoid some explanations.

\section{Edit for order}

Given text is read from top to bottom and often includes figures, it is important to make sure they show up in a sensible order.
The common convention is to make sure that figures appear after they are mentioned in the text, and while editing, you should make sure that all figures included in your work are also referred by your text.

\section{Edit for clarity and readability}

Besides the reasons I mentioned in the introduction, I tend to edit for clarity to keep my message clear and understandable.
Although scientific text tends to be heavy to read occasionally, that does not mean we should not strive to keep it accessible to understand the main results at least.

\section{Edit to meet page limits}

Often you have to comply with the page limits given by your publication target.
For example, Bachelor's theses might have a preferred page count or a conference might require you to fit your content to a certain amount of pages while using their template.
Depending on the topic and your writing style, condensing your content to fit a limit is an art by itself.
The challenge is in being able to include enough information to keep the content approachable while getting your main points through.

Besides maintaining a high-level view on what is important to include, there are macro-level tweaks you can perform to hit a page limit.
Often simple rephrasing or omitting unnecessary words (for example "very") can be useful and at times there are other ways to simplify content.
In case you refer to web sources, occasionally titles of referred sites can be cleaned up without losing anything essential to with a row or two as sometimes that is all you need to eliminate a page.
Editing for length is an art in itself and occasionally it may require a degree of creativity as you rewrite portions of your work.
When rewriting, it is still a good idea to keep the original passages around within a comment for example if your writing environment supports it.

It is good to consider that the type of your research has impact on page amount as qualitative work tends to require more space since it implies longer explanations for your study setup and results.
The point is that when you aim for a publication with strict page limits, you may have to condense your results a lot and it may even make sense to choose your methodology so that it suitable for short form.

\section{Conclusion}

Editing is an important part of a writing process as it allows you to take good content and turn it into great according to your desired metrics.
Often you are forced to edit your own work personally although in group collaboration this type of work happens in a more distributed manner.
Regardless it is a good idea to get used to editing your work and understanding the impact it can have.
